\documentclass[]{article}
\usepackage{lmodern}
\usepackage{amssymb,amsmath}
\usepackage{ifxetex,ifluatex}
\usepackage{fixltx2e} % provides \textsubscript
\ifnum 0\ifxetex 1\fi\ifluatex 1\fi=0 % if pdftex
  \usepackage[T1]{fontenc}
  \usepackage[utf8]{inputenc}
\else % if luatex or xelatex
  \ifxetex
    \usepackage{mathspec}
  \else
    \usepackage{fontspec}
  \fi
  \defaultfontfeatures{Ligatures=TeX,Scale=MatchLowercase}
\fi
% use upquote if available, for straight quotes in verbatim environments
\IfFileExists{upquote.sty}{\usepackage{upquote}}{}
% use microtype if available
\IfFileExists{microtype.sty}{%
\usepackage{microtype}
\UseMicrotypeSet[protrusion]{basicmath} % disable protrusion for tt fonts
}{}
\usepackage[margin=1in]{geometry}
\usepackage{hyperref}
\hypersetup{unicode=true,
            pdftitle={Final Exam - Rissler Energy Analysis},
            pdfauthor={Alex Rhomberg},
            pdfborder={0 0 0},
            breaklinks=true}
\urlstyle{same}  % don't use monospace font for urls
\usepackage{color}
\usepackage{fancyvrb}
\newcommand{\VerbBar}{|}
\newcommand{\VERB}{\Verb[commandchars=\\\{\}]}
\DefineVerbatimEnvironment{Highlighting}{Verbatim}{commandchars=\\\{\}}
% Add ',fontsize=\small' for more characters per line
\usepackage{framed}
\definecolor{shadecolor}{RGB}{248,248,248}
\newenvironment{Shaded}{\begin{snugshade}}{\end{snugshade}}
\newcommand{\AlertTok}[1]{\textcolor[rgb]{0.94,0.16,0.16}{#1}}
\newcommand{\AnnotationTok}[1]{\textcolor[rgb]{0.56,0.35,0.01}{\textbf{\textit{#1}}}}
\newcommand{\AttributeTok}[1]{\textcolor[rgb]{0.77,0.63,0.00}{#1}}
\newcommand{\BaseNTok}[1]{\textcolor[rgb]{0.00,0.00,0.81}{#1}}
\newcommand{\BuiltInTok}[1]{#1}
\newcommand{\CharTok}[1]{\textcolor[rgb]{0.31,0.60,0.02}{#1}}
\newcommand{\CommentTok}[1]{\textcolor[rgb]{0.56,0.35,0.01}{\textit{#1}}}
\newcommand{\CommentVarTok}[1]{\textcolor[rgb]{0.56,0.35,0.01}{\textbf{\textit{#1}}}}
\newcommand{\ConstantTok}[1]{\textcolor[rgb]{0.00,0.00,0.00}{#1}}
\newcommand{\ControlFlowTok}[1]{\textcolor[rgb]{0.13,0.29,0.53}{\textbf{#1}}}
\newcommand{\DataTypeTok}[1]{\textcolor[rgb]{0.13,0.29,0.53}{#1}}
\newcommand{\DecValTok}[1]{\textcolor[rgb]{0.00,0.00,0.81}{#1}}
\newcommand{\DocumentationTok}[1]{\textcolor[rgb]{0.56,0.35,0.01}{\textbf{\textit{#1}}}}
\newcommand{\ErrorTok}[1]{\textcolor[rgb]{0.64,0.00,0.00}{\textbf{#1}}}
\newcommand{\ExtensionTok}[1]{#1}
\newcommand{\FloatTok}[1]{\textcolor[rgb]{0.00,0.00,0.81}{#1}}
\newcommand{\FunctionTok}[1]{\textcolor[rgb]{0.00,0.00,0.00}{#1}}
\newcommand{\ImportTok}[1]{#1}
\newcommand{\InformationTok}[1]{\textcolor[rgb]{0.56,0.35,0.01}{\textbf{\textit{#1}}}}
\newcommand{\KeywordTok}[1]{\textcolor[rgb]{0.13,0.29,0.53}{\textbf{#1}}}
\newcommand{\NormalTok}[1]{#1}
\newcommand{\OperatorTok}[1]{\textcolor[rgb]{0.81,0.36,0.00}{\textbf{#1}}}
\newcommand{\OtherTok}[1]{\textcolor[rgb]{0.56,0.35,0.01}{#1}}
\newcommand{\PreprocessorTok}[1]{\textcolor[rgb]{0.56,0.35,0.01}{\textit{#1}}}
\newcommand{\RegionMarkerTok}[1]{#1}
\newcommand{\SpecialCharTok}[1]{\textcolor[rgb]{0.00,0.00,0.00}{#1}}
\newcommand{\SpecialStringTok}[1]{\textcolor[rgb]{0.31,0.60,0.02}{#1}}
\newcommand{\StringTok}[1]{\textcolor[rgb]{0.31,0.60,0.02}{#1}}
\newcommand{\VariableTok}[1]{\textcolor[rgb]{0.00,0.00,0.00}{#1}}
\newcommand{\VerbatimStringTok}[1]{\textcolor[rgb]{0.31,0.60,0.02}{#1}}
\newcommand{\WarningTok}[1]{\textcolor[rgb]{0.56,0.35,0.01}{\textbf{\textit{#1}}}}
\usepackage{graphicx,grffile}
\makeatletter
\def\maxwidth{\ifdim\Gin@nat@width>\linewidth\linewidth\else\Gin@nat@width\fi}
\def\maxheight{\ifdim\Gin@nat@height>\textheight\textheight\else\Gin@nat@height\fi}
\makeatother
% Scale images if necessary, so that they will not overflow the page
% margins by default, and it is still possible to overwrite the defaults
% using explicit options in \includegraphics[width, height, ...]{}
\setkeys{Gin}{width=\maxwidth,height=\maxheight,keepaspectratio}
\IfFileExists{parskip.sty}{%
\usepackage{parskip}
}{% else
\setlength{\parindent}{0pt}
\setlength{\parskip}{6pt plus 2pt minus 1pt}
}
\setlength{\emergencystretch}{3em}  % prevent overfull lines
\providecommand{\tightlist}{%
  \setlength{\itemsep}{0pt}\setlength{\parskip}{0pt}}
\setcounter{secnumdepth}{0}
% Redefines (sub)paragraphs to behave more like sections
\ifx\paragraph\undefined\else
\let\oldparagraph\paragraph
\renewcommand{\paragraph}[1]{\oldparagraph{#1}\mbox{}}
\fi
\ifx\subparagraph\undefined\else
\let\oldsubparagraph\subparagraph
\renewcommand{\subparagraph}[1]{\oldsubparagraph{#1}\mbox{}}
\fi

%%% Use protect on footnotes to avoid problems with footnotes in titles
\let\rmarkdownfootnote\footnote%
\def\footnote{\protect\rmarkdownfootnote}

%%% Change title format to be more compact
\usepackage{titling}

% Create subtitle command for use in maketitle
\providecommand{\subtitle}[1]{
  \posttitle{
    \begin{center}\large#1\end{center}
    }
}

\setlength{\droptitle}{-2em}

  \title{Final Exam - Rissler Energy Analysis}
    \pretitle{\vspace{\droptitle}\centering\huge}
  \posttitle{\par}
    \author{Alex Rhomberg}
    \preauthor{\centering\large\emph}
  \postauthor{\par}
      \predate{\centering\large\emph}
  \postdate{\par}
    \date{12/12/2019}


\begin{document}
\maketitle

\hypertarget{house-energy-trends}{%
\subsection{House Energy Trends}\label{house-energy-trends}}

\begin{Shaded}
\begin{Highlighting}[]
\NormalTok{FromAlliantConsumptionRaw }\OperatorTok\StringTok{ }\KeywordTok{filter}\NormalTok{(STARTTIME }\OperatorTok{!=}\StringTok{ }\KeywordTok{is.na}\NormalTok{(STARTTIME)) }\OperatorTok\StringTok{ }\KeywordTok{group_by}\NormalTok{(year, month) }\OperatorTok\StringTok{ }
\StringTok{  }\KeywordTok{summarize}\NormalTok{(}\DataTypeTok{Usage =} \KeywordTok{sum}\NormalTok{(USAGE)) }\OperatorTok\StringTok{ }\KeywordTok{gf_point}\NormalTok{(Usage }\OperatorTok{~}\StringTok{ }\NormalTok{month}\OperatorTok{|}\StringTok{ }\NormalTok{year) }\OperatorTok\StringTok{ }\KeywordTok{gf_labs}\NormalTok{(}\DataTypeTok{x=}\StringTok{"Month of Year"}\NormalTok{, }\DataTypeTok{y =} \StringTok{"Usage"}\NormalTok{)}
\end{Highlighting}
\end{Shaded}

\includegraphics{FinalExam_files/figure-latex/Usage by Month-1.pdf}

\begin{Shaded}
\begin{Highlighting}[]
\NormalTok{WeatherRaw }\OperatorTok\StringTok{ }\KeywordTok{group_by}\NormalTok{(year, month) }\OperatorTok\StringTok{ }\KeywordTok{summarize}\NormalTok{(}\DataTypeTok{temp =} \KeywordTok{mean}\NormalTok{(apparentTemperature)) }\OperatorTok\StringTok{ }\KeywordTok{gf_point}\NormalTok{(temp }\OperatorTok{~}\StringTok{ }\NormalTok{month }\OperatorTok{|}\StringTok{ }\NormalTok{year) }\OperatorTok\StringTok{ }\KeywordTok{gf_labs}\NormalTok{(}\DataTypeTok{x=}\StringTok{"Month of Year"}\NormalTok{, }\DataTypeTok{y =} \StringTok{"Apparent Temp Avg"}\NormalTok{)}
\end{Highlighting}
\end{Shaded}

\begin{verbatim}
## Warning: Removed 1 rows containing missing values (geom_point).
\end{verbatim}

\includegraphics{FinalExam_files/figure-latex/Weather by Month-1.pdf}

The first graph above shows monthly energy usage KWH over the span of
2018 and 2019. The second graph shows the apparent temperature F over
the same duration. At first look, there does not seem to be much
relation between the two graphs, but it can be seen that when weather
spikes at a high or a low, energy usage is almost always at a high peak,
showing more extreme temperatures resulted in greater energy usage.

\begin{Shaded}
\begin{Highlighting}[]
\NormalTok{FromAlliantConsumptionRaw }\OperatorTok\StringTok{ }\KeywordTok{filter}\NormalTok{(STARTTIME }\OperatorTok{!=}\StringTok{ }\KeywordTok{is.na}\NormalTok{(STARTTIME)) }\OperatorTok\StringTok{ }\KeywordTok{group_by}\NormalTok{(month, day) }\OperatorTok\StringTok{ }
\StringTok{  }\KeywordTok{summarize}\NormalTok{(}\DataTypeTok{Usage =} \KeywordTok{sum}\NormalTok{(USAGE)) }\OperatorTok\StringTok{ }\KeywordTok{gf_point}\NormalTok{(Usage }\OperatorTok{~}\StringTok{ }\NormalTok{day }\OperatorTok{|}\StringTok{ }\NormalTok{month) }\OperatorTok\StringTok{ }\KeywordTok{gf_labs}\NormalTok{(}\DataTypeTok{x=}\StringTok{"Day of Month"}\NormalTok{, }\DataTypeTok{y =} \StringTok{"Usage"}\NormalTok{)}
\end{Highlighting}
\end{Shaded}

\includegraphics{FinalExam_files/figure-latex/Usage by days of the month for each month-1.pdf}

\begin{Shaded}
\begin{Highlighting}[]
\NormalTok{WeatherRaw }\OperatorTok\StringTok{ }\KeywordTok{group_by}\NormalTok{(month, day) }\OperatorTok\StringTok{ }\KeywordTok{summarise}\NormalTok{(}\DataTypeTok{temp =} \KeywordTok{mean}\NormalTok{(apparentTemperature)) }\OperatorTok\StringTok{ }\KeywordTok{gf_point}\NormalTok{(temp }\OperatorTok{~}\StringTok{ }\NormalTok{day }\OperatorTok{|}\StringTok{ }\NormalTok{month) }\OperatorTok\StringTok{ }\KeywordTok{gf_labs}\NormalTok{(}\DataTypeTok{x=}\StringTok{"Month"}\NormalTok{, }\DataTypeTok{y =} \StringTok{"Apparent Temp Avg"}\NormalTok{)}
\end{Highlighting}
\end{Shaded}

\begin{verbatim}
## Warning: Removed 1 rows containing missing values (geom_point).
\end{verbatim}

\includegraphics{FinalExam_files/figure-latex/Weather by days of the month for each month-1.pdf}

Here we have a similar display of energy usage KWH and apparent
temperature F over the days of the month for the 12 months. Most of the
energy usage stays the same throughout the year until the warm summer
months are hit, then the energy usage increases greatly. The respective
months in the second graph show very consistently high tempertatures,
alluding that most probably Rissler does not like his house to be warm
in the summer.

\begin{Shaded}
\begin{Highlighting}[]
\NormalTok{FromAlliantConsumptionRaw }\OperatorTok\StringTok{ }\KeywordTok{filter}\NormalTok{(STARTTIME }\OperatorTok{!=}\StringTok{ }\KeywordTok{is.na}\NormalTok{(STARTTIME)) }\OperatorTok\StringTok{ }\KeywordTok{group_by}\NormalTok{(hour, minute) }\OperatorTok\StringTok{ }
\StringTok{  }\KeywordTok{summarize}\NormalTok{(}\DataTypeTok{MonthUsage =} \KeywordTok{sum}\NormalTok{(USAGE)) }\OperatorTok\StringTok{ }\KeywordTok{gf_point}\NormalTok{(MonthUsage }\OperatorTok{~}\StringTok{ }\NormalTok{hour ) }\OperatorTok\StringTok{ }\KeywordTok{gf_labs}\NormalTok{(}\DataTypeTok{x=}\StringTok{"Hour"}\NormalTok{, }\DataTypeTok{y =} \StringTok{"Usage"}\NormalTok{)}
\end{Highlighting}
\end{Shaded}

\includegraphics{FinalExam_files/figure-latex/Usage by hour of the day-1.pdf}

\begin{Shaded}
\begin{Highlighting}[]
\NormalTok{WeatherRaw }\OperatorTok\StringTok{ }\KeywordTok{group_by}\NormalTok{(hour, minute) }\OperatorTok\StringTok{ }\KeywordTok{summarise}\NormalTok{(}\DataTypeTok{temp =} \KeywordTok{mean}\NormalTok{(apparentTemperature)) }\OperatorTok\StringTok{ }\KeywordTok{gf_point}\NormalTok{(temp }\OperatorTok{~}\StringTok{ }\NormalTok{hour) }\OperatorTok\StringTok{ }\KeywordTok{gf_labs}\NormalTok{(}\DataTypeTok{x=}\StringTok{"Hour"}\NormalTok{, }\DataTypeTok{y =} \StringTok{"Apparent Temp Avg"}\NormalTok{)}
\end{Highlighting}
\end{Shaded}

\begin{verbatim}
## Warning: Removed 1 rows containing missing values (geom_point).
\end{verbatim}

\includegraphics{FinalExam_files/figure-latex/Weather by hour of the day-1.pdf}

These two graphs show the energy usage HWK and apparent temperature F
over the 24 hours in a day. The graphs follow similar curves, but are
off set from each other in time. Energy usage begins to increase as
temperature starts to decrease from its peak and vice versa.

\begin{Shaded}
\begin{Highlighting}[]
\NormalTok{FromAlliantConsumptionRaw }\OperatorTok\StringTok{ }\KeywordTok{filter}\NormalTok{(STARTTIME }\OperatorTok{!=}\StringTok{ }\KeywordTok{is.na}\NormalTok{(STARTTIME)) }\OperatorTok\StringTok{ }\KeywordTok{group_by}\NormalTok{(day, hour) }\OperatorTok\StringTok{ }
\StringTok{  }\KeywordTok{summarize}\NormalTok{(}\DataTypeTok{MonthUsage =} \KeywordTok{sum}\NormalTok{(USAGE)) }\OperatorTok\StringTok{ }\KeywordTok{gf_point}\NormalTok{(MonthUsage }\OperatorTok{~}\StringTok{ }\NormalTok{hour }\OperatorTok{|}\StringTok{ }\NormalTok{day) }\OperatorTok\StringTok{ }\KeywordTok{gf_labs}\NormalTok{(}\DataTypeTok{x=}\StringTok{"Hour"}\NormalTok{, }\DataTypeTok{y =} \StringTok{"Usage"}\NormalTok{)}
\end{Highlighting}
\end{Shaded}

\includegraphics{FinalExam_files/figure-latex/Usage of each hour of the day for each day of the month-1.pdf}

\begin{Shaded}
\begin{Highlighting}[]
\NormalTok{WeatherRaw }\OperatorTok\StringTok{ }\KeywordTok{group_by}\NormalTok{(day, hour) }\OperatorTok\StringTok{ }\KeywordTok{summarise}\NormalTok{(}\DataTypeTok{temp =} \KeywordTok{mean}\NormalTok{(apparentTemperature)) }\OperatorTok\StringTok{ }\KeywordTok{gf_point}\NormalTok{(temp }\OperatorTok{~}\StringTok{ }\NormalTok{hour }\OperatorTok{|}\StringTok{ }\NormalTok{day) }\OperatorTok\StringTok{ }\KeywordTok{gf_labs}\NormalTok{(}\DataTypeTok{x=}\StringTok{"Hour"}\NormalTok{, }\DataTypeTok{y =} \StringTok{"Apparent Temp Avg"}\NormalTok{)}
\end{Highlighting}
\end{Shaded}

\begin{verbatim}
## Warning: Removed 1 rows containing missing values (geom_point).
\end{verbatim}

\includegraphics{FinalExam_files/figure-latex/Weather of each hour of the day for each day of the month-1.pdf}

These two graphs show the energy usage KWH and apparent temperature F
over the hours of the day with each day of the month marked. These
graphs follow similar trends as the one above, it is just interesting to
see how various days throughout the month differ from each other.

\hypertarget{electric-heat-pump-water-heater-switch}{%
\subsection{Electric Heat Pump Water Heater
Switch}\label{electric-heat-pump-water-heater-switch}}

\begin{Shaded}
\begin{Highlighting}[]
\NormalTok{Rissler_}\DecValTok{955}\NormalTok{_Kirkwood_St }\OperatorTok\StringTok{ }\KeywordTok{filter}\NormalTok{(WaterHeater }\OperatorTok{==}\StringTok{ }\DecValTok{0}\NormalTok{) }\OperatorTok\StringTok{ }\KeywordTok{gf_line}\NormalTok{(TotalUsage }\OperatorTok{~}\StringTok{ }\NormalTok{Month, }\DataTypeTok{color =} \StringTok{"Red"}\NormalTok{) }\OperatorTok\StringTok{ }\KeywordTok{gf_line}\NormalTok{(Bill }\OperatorTok{~}\StringTok{ }\NormalTok{Month, }\DataTypeTok{color =} \StringTok{"Green"}\NormalTok{) }\OperatorTok\StringTok{ }\KeywordTok{gf_theme}\NormalTok{(}\DataTypeTok{axis.text.x =} \KeywordTok{element_text}\NormalTok{(}\DataTypeTok{angle =} \DecValTok{90}\NormalTok{, }\DataTypeTok{hjust =} \DecValTok{1}\NormalTok{)) }\OperatorTok\StringTok{ }\KeywordTok{gf_labs}\NormalTok{(}\DataTypeTok{x=}\StringTok{"Time"}\NormalTok{, }\DataTypeTok{y =} \StringTok{"Usage"}\NormalTok{)}
\end{Highlighting}
\end{Shaded}

\includegraphics{FinalExam_files/figure-latex/Usage before Water Heater-1.pdf}

\begin{Shaded}
\begin{Highlighting}[]
\NormalTok{Rissler_}\DecValTok{955}\NormalTok{_Kirkwood_St }\OperatorTok\StringTok{ }\KeywordTok{filter}\NormalTok{(WaterHeater }\OperatorTok{==}\StringTok{ }\DecValTok{1}\NormalTok{) }\OperatorTok\StringTok{ }\KeywordTok{gf_line}\NormalTok{(TotalUsage }\OperatorTok{~}\StringTok{ }\NormalTok{Month, }\DataTypeTok{color =} \StringTok{"Red"}\NormalTok{) }\OperatorTok\StringTok{ }\KeywordTok{gf_line}\NormalTok{(Bill }\OperatorTok{~}\StringTok{ }\NormalTok{Month, }\DataTypeTok{color =} \StringTok{"Green"}\NormalTok{) }\OperatorTok\StringTok{ }\KeywordTok{gf_theme}\NormalTok{(}\DataTypeTok{axis.text.x =} \KeywordTok{element_text}\NormalTok{(}\DataTypeTok{angle =} \DecValTok{90}\NormalTok{, }\DataTypeTok{hjust =} \DecValTok{1}\NormalTok{)) }\OperatorTok\StringTok{ }\KeywordTok{gf_labs}\NormalTok{(}\DataTypeTok{x=}\StringTok{"Time"}\NormalTok{, }\DataTypeTok{y =} \StringTok{"Usage"}\NormalTok{)}
\end{Highlighting}
\end{Shaded}

\includegraphics{FinalExam_files/figure-latex/Usage After Heater-1.pdf}

\begin{Shaded}
\begin{Highlighting}[]
\NormalTok{Rissler_}\DecValTok{955}\NormalTok{_Kirkwood_St }\OperatorTok\StringTok{ }\KeywordTok{group_by}\NormalTok{(WaterHeater) }\OperatorTok\StringTok{ }\KeywordTok{summarise}\NormalTok{(}\DataTypeTok{UsageMean =} \KeywordTok{mean}\NormalTok{(TotalUsage), }\DataTypeTok{BillMean =} \KeywordTok{mean}\NormalTok{(Bill))}
\end{Highlighting}
\end{Shaded}

\begin{verbatim}
## # A tibble: 2 x 3
##   WaterHeater UsageMean BillMean
##         <dbl>     <dbl>    <dbl>
## 1           0      509.    101. 
## 2           1      462.     46.0
\end{verbatim}

In short, making the switch to an electric heat pump water heater was
benneficial. The first graph shows the energy usage KWH in Red over time
and the bill price in Green, and the second graph shows the same thing
but after the electric heat pump water heater was installed. The energy
usage before installing sits around the 300 - 600 KWH usage with a bill
average close to 100. After installation, the usage range drops to 200 -
400 on average with a bill well below 100, averaging around 25 - 50.

\hypertarget{enphase-vs-alliant}{%
\subsection{Enphase vs Alliant}\label{enphase-vs-alliant}}

\begin{Shaded}
\begin{Highlighting}[]
\NormalTok{FromAlliantConsumptionRaw }\OperatorTok\StringTok{ }\KeywordTok{filter}\NormalTok{(STARTTIME }\OperatorTok{!=}\StringTok{ }\KeywordTok{is.na}\NormalTok{(STARTTIME), year }\OperatorTok{==}\StringTok{ }\DecValTok{2019}\NormalTok{) }\OperatorTok\StringTok{ }\KeywordTok{group_by}\NormalTok{(year, month) }\OperatorTok\StringTok{ }
\StringTok{  }\KeywordTok{summarize}\NormalTok{(}\DataTypeTok{Usage =} \KeywordTok{sum}\NormalTok{(USAGE)) }\OperatorTok\StringTok{ }\KeywordTok{gf_point}\NormalTok{(Usage }\OperatorTok{~}\StringTok{ }\NormalTok{month}\OperatorTok{|}\StringTok{ }\NormalTok{year) }\OperatorTok\StringTok{ }\KeywordTok{gf_labs}\NormalTok{(}\DataTypeTok{x=}\StringTok{"Month of Year"}\NormalTok{, }\DataTypeTok{y =} \StringTok{"Usage"}\NormalTok{)}
\end{Highlighting}
\end{Shaded}

\includegraphics{FinalExam_files/figure-latex/Usage by Month vs Enphase-1.pdf}

\begin{Shaded}
\begin{Highlighting}[]
\NormalTok{EnphaseRaw }\OperatorTok\StringTok{ }\KeywordTok{filter}\NormalTok{(DATETIME }\OperatorTok{!=}\StringTok{ }\KeywordTok{is.na}\NormalTok{(DATETIME), year }\OperatorTok{==}\StringTok{ }\DecValTok{2019}\NormalTok{) }\OperatorTok\StringTok{ }\KeywordTok{group_by}\NormalTok{(year, month) }\OperatorTok\StringTok{ }\KeywordTok{summarize}\NormalTok{(}\DataTypeTok{Usage =} \KeywordTok{sum}\NormalTok{(EnergyConsumed)) }\OperatorTok\StringTok{ }\KeywordTok{gf_point}\NormalTok{(Usage }\OperatorTok{~}\StringTok{ }\NormalTok{month }\OperatorTok{|}\StringTok{ }\NormalTok{year) }\OperatorTok\StringTok{ }\KeywordTok{gf_labs}\NormalTok{(}\DataTypeTok{x=}\StringTok{"Month of Year"}\NormalTok{, }\DataTypeTok{y =} \StringTok{"Usage"}\NormalTok{)}
\end{Highlighting}
\end{Shaded}

\includegraphics{FinalExam_files/figure-latex/Enphase Usage by Month-1.pdf}

The first graph above is from the first analysis of this document and
shows the energy usage KWH of the months of the year with the data from
Alliant. The second graph shows the energy usage KWH from the enphase
device on Rissler's house. It can be seen that the two follow almost
identical trends, meaning Alliant is producing accurate records in
allign with the enphase device.

\begin{Shaded}
\begin{Highlighting}[]
\NormalTok{FromAlliantConsumptionRaw }\OperatorTok\StringTok{ }\KeywordTok{filter}\NormalTok{(STARTTIME }\OperatorTok{!=}\StringTok{ }\KeywordTok{is.na}\NormalTok{(STARTTIME)) }\OperatorTok\StringTok{ }\KeywordTok{group_by}\NormalTok{(hour, minute) }\OperatorTok\StringTok{ }
\StringTok{  }\KeywordTok{summarize}\NormalTok{(}\DataTypeTok{MonthUsage =} \KeywordTok{sum}\NormalTok{(USAGE)) }\OperatorTok\StringTok{ }\KeywordTok{gf_point}\NormalTok{(MonthUsage }\OperatorTok{~}\StringTok{ }\NormalTok{hour ) }\OperatorTok\StringTok{ }\KeywordTok{gf_labs}\NormalTok{(}\DataTypeTok{x=}\StringTok{"Hour"}\NormalTok{, }\DataTypeTok{y =} \StringTok{"Usage"}\NormalTok{)}
\end{Highlighting}
\end{Shaded}

\includegraphics{FinalExam_files/figure-latex/Usage by hour of the day vs Enphase-1.pdf}

Similarly, the first graph from above is taken from the inital analysis
on this document and shows energy usage KWH over the hours of the day
with information from Alliant. The second graph shows the same thing but
from the enphase device. Both graphs follow similar trends but are not
exactly accurate, showing there could be a margin of error with the
enphase device, or that the information from Alliant is wrong.

\begin{Shaded}
\begin{Highlighting}[]
\NormalTok{EnphaseRaw }\OperatorTok\StringTok{ }\KeywordTok{filter}\NormalTok{(DATETIME }\OperatorTok{!=}\StringTok{ }\KeywordTok{is.na}\NormalTok{(DATETIME)) }\OperatorTok\StringTok{ }\KeywordTok{group_by}\NormalTok{(hour, minute) }\OperatorTok\StringTok{ }\KeywordTok{summarize}\NormalTok{(}\DataTypeTok{MonthUsage =} \KeywordTok{sum}\NormalTok{(EnergyConsumed)) }\OperatorTok\StringTok{ }\KeywordTok{gf_point}\NormalTok{(MonthUsage }\OperatorTok{~}\StringTok{ }\NormalTok{hour) }\OperatorTok\StringTok{ }\KeywordTok{gf_labs}\NormalTok{(}\DataTypeTok{x=}\StringTok{"Hour"}\NormalTok{, }\DataTypeTok{y =} \StringTok{"Usage"}\NormalTok{)}
\end{Highlighting}
\end{Shaded}

\includegraphics{FinalExam_files/figure-latex/Enphase usage by hour of the day-1.pdf}

\hypertarget{consistency-of-bills}{%
\subsection{Consistency of Bills}\label{consistency-of-bills}}

\begin{Shaded}
\begin{Highlighting}[]
\NormalTok{Rissler_}\DecValTok{955}\NormalTok{_Kirkwood_St }\OperatorTok\StringTok{ }\KeywordTok{gf_line}\NormalTok{(Bill }\OperatorTok{~}\StringTok{ }\NormalTok{Month, }\DataTypeTok{color =} \StringTok{"Purple"}\NormalTok{) }\OperatorTok\StringTok{ }\KeywordTok{gf_line}\NormalTok{(TotalUsage }\OperatorTok{~}\StringTok{ }\NormalTok{Month, }\DataTypeTok{color =} \StringTok{"Blue"}\NormalTok{) }\OperatorTok\StringTok{ }\KeywordTok{gf_line}\NormalTok{(UsageGrid }\OperatorTok{~}\StringTok{ }\NormalTok{Month, }\DataTypeTok{color =} \StringTok{"Red"}\NormalTok{) }\OperatorTok\StringTok{ }\KeywordTok{gf_line}\NormalTok{(ProductionToGrid }\OperatorTok{~}\StringTok{ }\NormalTok{Month, }\DataTypeTok{color =} \StringTok{"Green"}\NormalTok{) }\OperatorTok\StringTok{ }\KeywordTok{gf_labs}\NormalTok{(}\DataTypeTok{x=}\StringTok{"Multivariate"}\NormalTok{, }\DataTypeTok{y =} \StringTok{"Time"}\NormalTok{)}
\end{Highlighting}
\end{Shaded}

\begin{verbatim}
## Warning: Removed 21 rows containing missing values (geom_path).
\end{verbatim}

\includegraphics{FinalExam_files/figure-latex/Bill vs Total Usage vs Grid Usage vs Grid Production-1.pdf}

\begin{Shaded}
\begin{Highlighting}[]
\NormalTok{WeatherRaw2 }\OperatorTok\StringTok{ }\KeywordTok{gf_line}\NormalTok{(apparentTemperature }\OperatorTok{~}\StringTok{ }\NormalTok{time) }\OperatorTok\StringTok{ }\KeywordTok{gf_labs}\NormalTok{(}\DataTypeTok{x=}\StringTok{"Time"}\NormalTok{, }\DataTypeTok{y =} \StringTok{"Temperature"}\NormalTok{)}
\end{Highlighting}
\end{Shaded}

\includegraphics{FinalExam_files/figure-latex/Weather Overview vs Bill-1.pdf}

The first graph above shows many different aspects of the bill and
energy usages over time. The Red line shows energy usage KWH with a
respective purple line representing the bill price. The blue line shows
the usage from the grid after solar and water heat pump installation
with the green line showing solar production to the grid. The second
graph shows the apparent temperatures during the same duration of time.

Before the installations in 2018, it is easy to follow the trends of the
the energy usage and the price of the bill. They remain consistent to
each other for the most part, showing similar humps and dips. It is a
little harder to read after the installations. Thinking of it
numerically, adding the red and green lines would equal the blue line.
When looking at the bill, if the green line is lower, the bill should be
higher and vice versa, which can be seen as true in the plot. Overall,
the bill seems to be consistent with the energery usage ammounts
reported by Alliant. When comparing to weather, it is interesting to see
that when the weather drops well below 0, it is almost a gaurantee the
energy bill and usage will spike high same with temperatures above 90, a
similar trend that can be seen in the first couple of graphs on this
ananlysis document.

\hypertarget{time-of-day-pricing-good}{%
\subsection{Time of Day Pricing, Good?}\label{time-of-day-pricing-good}}

\begin{Shaded}
\begin{Highlighting}[]
\NormalTok{FromAlliantConsumptionRaw }\OperatorTok\StringTok{ }\KeywordTok{filter}\NormalTok{(STARTTIME }\OperatorTok{!=}\StringTok{ }\KeywordTok{is.na}\NormalTok{(STARTTIME)) }\OperatorTok\StringTok{ }\KeywordTok{group_by}\NormalTok{(hour, minute) }\OperatorTok\StringTok{ }
\StringTok{  }\KeywordTok{summarize}\NormalTok{(}\DataTypeTok{MonthUsage =} \KeywordTok{sum}\NormalTok{(USAGE)) }\OperatorTok\StringTok{ }\KeywordTok{gf_point}\NormalTok{(MonthUsage }\OperatorTok{~}\StringTok{ }\NormalTok{hour ) }\OperatorTok\StringTok{ }\KeywordTok{gf_labs}\NormalTok{(}\DataTypeTok{x=}\StringTok{"Hour"}\NormalTok{, }\DataTypeTok{y =} \StringTok{"Usage"}\NormalTok{)}
\end{Highlighting}
\end{Shaded}

\includegraphics{FinalExam_files/figure-latex/Usage by hour of the day 2-1.pdf}

The graph above shows the monthly usage on the y axis with the hour of
the day on the x axis.

Most of the energy usage takes place from 3:00 pm to 6:00 am. With the
time of day prices during the hours of 8:00 pm to 7:00 am being close to
half off in both seasons, switching to this plan makes the most amount
of sense. Seventy Five percent of the high energy usage takes place when
the price of usage is close to half off, so a significant amount of
money is saved because of this. Basically, a 50\% coupon for those
times. Making this switch is a smart decision if most of your energy
usage falls between the discounted hours, not vice versa.


\end{document}
